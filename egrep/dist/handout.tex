\documentclass{article}

% Fonts
\usepackage{fontspec}

% General formatting
\usepackage[margin=1.3in, headheight=3ex, headsep=3ex]{geometry}
\usepackage{fancyhdr}
\usepackage{parskip}
\usepackage{amsmath}

% Colour
\usepackage[usenames,dvipsnames]{xcolor}

% Hyperlinks
\usepackage{url}
\usepackage{hyperref}

% Course details
\newcommand{\longcoursename}{
    Student Taught Courses (StuCo): Shilling the Rust Programming Language
}

\newcommand{\shortcoursename}{STUCO: RUSTLANG}
\newcommand{\courselocation}{PH A18B}
\newcommand{\meetingstarttime}{18:40}
\newcommand{\meetingendtime}{19:30}
\newcommand{\meetingdays}{Wednesday}
\newcommand{\longsemester}{Fall 2022}
\newcommand{\shortsemester}{F22}
\newcommand{\academicyear}{2021-22}
\newcommand{\deptcode}{98}
\newcommand{\coursecode}{008}
\newcommand{\fullcoursecode}{\deptcode-\coursecode}

% Headers and Footers
\pagestyle{fancy}
\lhead{Pierce; Duvall}
\rhead{\fullcoursecode\ \shortsemester}

\begin{document}
\thispagestyle{empty}
\begin{center}
    \begin{minipage}{.85\textwidth}
        \centering
        {\huge {\fullcoursecode} Homework 3: \texttt{egrep}}

        \vspace{1em}

        \begin{tabular}{@{}rl@{}}
            Cooper Pierce & \href{mailto:cppierce@andrew.cmu.edu}{\texttt{cppierce@andrew.cmu.edu}} \\
            Jack Duvall & \href{mailto:jrduvall@andrew.cmu.edu}{\texttt{jrduvall@andrew.cmu.edu}} \\
        \end{tabular}

        \vspace{1em}

        \longsemester
    \end{minipage}
\end{center}

\section*{Overview}

The goal of assignment is to apply what we've talked thus far in class, and
start to apply it to a larger problem, with some more involved existing code. It
also involves IO, which you'll likely use in your project, as well as applying
some of the topics, like traits, that we've discussed. A good understanding of
the standard library will also help! Documentation for the standard library can
be found at \url{https://doc.rust-lang.org/std/} and is an invaluable resource.

\section*{\texttt{egrep}}

In this assignment you'll be implementing components of the utility
\texttt{egrep} (this is the same as \texttt{grep -E}, but aliased as one
command---the regex format is a bit more sane than by default). Most of your
work will be to utilise the existing NFA based regex matching engine to actually
interact with command line input, and producing the corresponding output.

\subsection*{Basic Matching} 

To start with, you'll need to implement the logic in \texttt{main} and some
parts of \texttt{Matcher} to be able to find matches in the first place. No need
to worry about parsing patterns or writing an efficient matching engine---we've
done that for you! We suggest you write a function like

\begin{verbatim}
fn get_files<'a>(
    files: impl Iterator<Item = &'a str>,
) -> Vec<io::Result<(&'a str, Box<dyn BufRead>)>> {
    todo!()
}
\end{verbatim}

In order to handle the input files, because we need to handle standard in.

If we're ever given a file name of \texttt{-}, or no files are provided, we want
to read input from standard in instead. Other inputs after the pattern should be
treated as relative paths, and you can read from these directly. In the case of
non-existent files or other IO related errors, you can print any reasonable
diagnostic, but you should exit with a non-zero exit code.

You will probably find
\href{https://doc.rust-lang.org/std/fs/struct.File.html}{\texttt{File}} and
\href{https://doc.rust-lang.org/std/io/fn.stdin.html}{\texttt{stdin}} useful.
Likewise, the
\href{https://doc.rust-lang.org/std/io/trait.Read.html}{\texttt{Read}} and
\href{https://doc.rust-lang.org/std/io/trait.Write.html}{\texttt{Write}} traits
from the standard library (or their buffered versions) are likely to be helpful
reading.

Your program should essentially behave the same as your system's install (sorry,
Windows users; perhaps consult andrew) of \texttt{egrep}. Specifically:

\begin{itemize}
    \item Each matching line, and only these lines, should be written to
        standard output
    \item If more than one file argument appears, prefix each such line with the
        filename followed by a colon
\end{itemize}

\clearpage

So if we ran \texttt{egrep foo A B} and both A and B contained one line each
consisting of foo, then we'd expect to see 

\begin{verbatim}
A:foo
B:foo
\end{verbatim}

Note that there's no space after the colon! You might not care but Gradescope
does.

You can assign standard input any reasonable name for the purpose of printing
its filename.

\subsection*{Highlighting Matches}

As an extension, now highlight the specific matching region! This involves a
little bit more implementation in the \texttt{Matcher}, but as before, most of
the work has already been done.

How you handle the printing here is up to you; the only requirement is that the
match is bolded and a different colour. We suggest the
\href{termion}{https://docs.rs/termion/latest/termion/} crate, but you can use
any crate you wish, or even handle the ANSI escape codes by hand (a bit ugly,
but it works). Note that Gradescope is a Linux environment, so whatever you do
has to work for this, so if you're on Windows don't use something Windows only.

What you print should roughly correspond to the results for \texttt{egrep
--color}, but as this is less standard, it doesn't need to be exact.

\section*{Submitting}

Submission will be on Gradescope.

You should submit a zip file containing the whole crate rooted at the directory
where \texttt{Cargo.toml} appears (e.g., run \texttt{zip submission.zip Cargo.toml
src/*}) and upload that.

\end{document}
